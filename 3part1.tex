\begin{document}
\color{white}
\pagecolor{black_background}

\section{Peso Unitario de el ladrillo 6 huecos}

Las especificaciones de ladrillo fueron extraidos de la pagina
\url{https://scz.tienda.incerpaz.com/producto/6her/} conocido como
ladrillo estandar (6 huecos), con dimensiones de 10x15x24 cm
y con un rendimiento de 24 pzas/m2 teniendo un peso de 2.65 kgf.
por cada ladrillo.

\section{peso 24 ladrillos 6 huecos}

$$24 \times 2.65 = 63.9 kgf.$$

\section{peso del mortero}

El peso especifico del mortero fue extraido de
\url{https://es.wikipedia.org/wiki/Anexo:Pesos_especifícos}

morteros de cemento portland y arena = 2100 kgf/m3

Como tenemos las dimensiones del ladrillo, lo que hacemos es descontar
el volumen del ladrillo al metro cuadrado de muro de ladrillo.

$$(1m \times 1m \times 0.1m) - (0.1m \times 0.15m \times 0.24m \times 24unid) = 0.0136m^3$$

Peso del mortero:

$$0.0136m^3 \times 2100 \frac{kgf}{m^3} = 28.56kgf$$

\section{Peso del reboque}

Considerando un reboque de 1.5 cm en ambas caras del muro se tiene un espesor
total de 3cm en un metro cuadrado de muro, por lo tanto:

$$2100 \frac{kgf}{m^3} \times 0.03m \times 1m \times 1m = 63kgf $$

\section{peso ladrillos + mortero + reboque}

Este peso es para un metro cuadrado de muro.

$$63.9kgf + 28.56kgf + 63kgf = 155.46kgf$$

\end{document}
